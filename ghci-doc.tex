
\documentclass{article}

\usepackage{listings}
\usepackage{ghci}

\usepackage{tabularx}


%include polycode.fmt

%format ghci4luatex = "\text{GHCi for Lua\TeX}"
%format luatex = "\text{Lua\TeX}"

\begin{ghci}
:set -XOverloadedStrings
\end{ghci}

\begin{ghci}
import Text.LaTeX
\end{ghci}


\begin{ghci}
x :: Int
x = 4

y :: Int
y = 5
\end{ghci}

\title{|ghci4luatex| \\ \vspace{0.4em}
  \large A ghci session in |luatex|}

\author{Alice Rixte}

\begin{document}

\maketitle

\section{Presentation}

\section{Getting started}

\section{The GHCi server}

\section{The GHCi package for |luatex|}

\section{Tips and tricks}

\hask{printTex (section "A section using HaTeX")}

The sum of $x$ and $y$ when $x = \hask{x}$ and $y = \hask{y}$ is $\hask{x + y}$.

\section{Combining lhs2TeX and ghci4luatex}



Let |sqr| be the square function:

\begin{code}
sqr n = n * n
\end{code}

\begin{ghci}
  sqr :: Num a => a -> a
  sqr n = n * n
\end{ghci}

Then  |sqr 3| is equal to \hask{sqr 3 :: Int}.


\end{document}
