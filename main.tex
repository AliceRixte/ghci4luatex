
\documentclass{article}

\usepackage{listings}
\usepackage{ghci}

\usepackage{tabularx}

%include polycode.fmt

\begin{ghci}
:set -XOverloadedStrings
\end{ghci}

\begin{ghci}
import Text.LaTeX

printTex = putStrLn . prettyLateX
\end{ghci}

\begin{ghci}
x :: Int
x = 5


y :: Int
y = 6
\end{ghci}

\begin{document}
\hask{printTex (section "A section using HaTeX")}

The sum of $x$ and $y$ when $x = \hask{x}$ and $y = \hask{y}$ is $\hask{x + y}$.

\section{Combining lhs2TeX and ghci4luatex}

\ghcisection{Combining lhs2Tex}

Let |sqr| be the square function:

\begin{code}
sqr n = n * n
\end{code}

\begin{ghci}
  sqr :: Num a => a -> a
  sqr n = n * n
\end{ghci}

Then  |sqr 3| is equal to \hask{sqr 3 :: Int}.


\section{Advanced : faster compilation with memoization}
\ghcisection{Using ghcisection}


Ignores any modification before this point.

Be careful, this means for instance that if you change the value of |x| in the code before, it will not be updated : $x = \hask{x}$.


\subsection{Back to main}
\ghcicontinue{main}
If you  want updates of |x| to take effect, you have to use \texttt{ghcicontinue}. The default section is called \texttt{main}.

Let's ask the value of |x| again : $x = \hask{x}$

\end{document}